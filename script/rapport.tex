\documentclass[a4paper,12pt]{article}
\usepackage[utf8]{inputenc}
\usepackage[T1]{fontenc}
\usepackage[francais]{babel}
\usepackage{caption}
\usepackage{graphicx}
\usepackage{setspace}
\usepackage{times}
\usepackage{tikz}
\usepackage[left=2cm,right=2cm,top=2cm,bottom=2cm]{geometry}
\usepackage[nottoc, notlof, notlot]{tocbibind}
%\usepackage[french]{algorithme}
\usepackage[square,numbers]{natbib}
\renewcommand{\baselinestretch}{1.5}
\setcounter{tocdepth}{4}
\title{Étude des séquences répétées centromériques de primates : fonction et évolution}
\author{Florence Jornod}
\date{\today}

\begin{document}
\maketitle
\thispagestyle{empty}
\newpage
\thispagestyle{empty}
\section*{Remerciements}
\newpage
\tableofcontents
\thispagestyle{empty}
\newpage
\setcounter{page}{1}
\section{Introduction}
\begin{itemize}
\item présentation des centromères en général
\begin{itemize}
\item ce que c'est
\item où c'est
\item à quoi ça sert
\end{itemize}
\item particularité chez les primates
\begin{itemize}
\item $\alpha$ sat 
\end{itemize}
\item interet de les étudier
\item pourquoi pas les methodes trad
\item presentation de ce qu'on fait
\begin{itemize}
\item developpement/amélioration d'une méthode 
\item application à 2 ou 3 espèces
\end{itemize}

\end{itemize}
\section{Matériel et Méthodes}
\subsection{Les données biologique} 
	\begin{itemize}
	\item especes étudiées
	\item d'où viennent les séquences
	\item ce qu'on en a fait ( = filtre + kmers)
	\end{itemize}
\subsection{L'algorithme/ le programme}
	\begin{itemize}
	\item algo/code préexistant
	\item pourquoi traduire et changer l'algo
	\item l'algo (ACP, distance(+ LDA), clustering, arret = matepair, taille max des familles)
	\item les modules utilisés
	\end{itemize}
\section{Résultats}
\begin{itemize}
\item param du matepair ( famille bleue si place)
\item param du nb de comp?
\item param de la lda ( validation via rand )
\item analyse des familles
\item taille et nombre des familles
\item taille des consensus
\end{itemize}
\section{Discussion}
\section{Conclusion}

\end{document}